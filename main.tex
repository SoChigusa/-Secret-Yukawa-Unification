\documentclass[12pt]{article}
% \pdfoutput=1
\usepackage{amsmath}
\usepackage{amssymb}
\usepackage{graphicx}
\usepackage{subcaption}
\usepackage{url}
\usepackage{comment}

%\usepackage{ifthen}
\usepackage[sort&compress, numbers, merge]{natbib}

    

\setlength{\textwidth}{17.9cm}
\setlength{\textheight}{23.0cm}
\setlength{\oddsidemargin}{-0.5cm}
\setlength{\evensidemargin}{0cm}
\setlength{\headheight}{0cm}
\setlength{\headsep}{0cm}
\setlength{\topmargin}{-0.5cm}
\setlength{\footskip}{1.5cm}



\newcommand{\Slash}[1]{{\ooalign{\hfil/\hfil\crcr$#1$}}}
\newcommand{\1}{\mbox{1}\hspace{-0.25em}\mbox{l}}


%%%%%%%%%%%%%%%%%%%%%%%%%%%%%%%
%%%    remove the following commands when finalizing
%%%%%%%%%%%%%%%%%%%%%%%%%%%%%%%
\def\TODO#1{ \textcolor{red}{\textbf{ ($\clubsuit$ #1 $\clubsuit$)} }}
%%%%%%%%%%%%%%%%%%%%%%%%%%%%%%%
%%%%%%%%%%%%%%%%%%%%%%%%%%%%%%%

\usepackage[colorlinks=true, linkcolor=blue, citecolor=blue,
urlcolor=black]{hyperref} 


\title{Yukawa Unification}
                 
\begin{document}
\baselineskip 0.6cm

\def\simgt{\mathrel{\lower2.5pt\vbox{\lineskip=0pt\baselineskip=0pt
           \hbox{$>$}\hbox{$\sim$}}}}
\def\simlt{\mathrel{\lower2.5pt\vbox{\lineskip=0pt\baselineskip=0pt
           \hbox{$<$}\hbox{$\sim$}}}}
\def\simprop{\mathrel{\lower3.0pt\vbox{\lineskip=1.0pt\baselineskip=0pt
             \hbox{$\propto$}\hbox{$\sim$}}}}
\def\tr{\mathop{\rm tr}}
\def\SU{\mathop{\rm SU}}

\begin{titlepage}

\begin{flushright}
IPMU18-00** \\
UT18
\end{flushright}

\vskip 1.1cm

\begin{center}

{\Large \bf 
Yukawa Unification in Split Supersymmetry
}

\vskip 1.2cm
So Chigusa$^1$,
Takeo Moroi$^{1,2}$,
Natsumi Nagata$^1$ and 
Satoshi Shirai$^2$
\vskip 0.5cm

{\it
{
$^1$
Department of Physics, Faculty of Science,
The University of Tokyo, Bunkyo-ku, Tokyo 113-0033, Japan
}\\
{$^2$Kavli Institute for the Physics and Mathematics of the Universe (WPI), \\The University of Tokyo Institutes for Advanced Study, The University of Tokyo, Kashiwa
 277-8583, Japan}
}


\vskip 1.0cm

\abstract{
We study bottom-tau Yukawa unification in the minimal supersymmetric Standard Model.
We focus on the parameter region where $\tan\beta \lesssim 1$ and found the Yukawa unification can be significantly improved.
}

\end{center}
\end{titlepage}

%%%%%%%%%%%%%%%%%%%%%%%%%%%%%%%%%%%%%%%%%%%%%%%%%%%%
\section{Introduction}
\label{sec:intro}
%%%%%%%%%%%%%%%%%%%%%%%%%%%%%%%%%%%%%%%%%%%%%%%%%%%%

%%%%%%%%%%%%%%%%%%%%%%%%%%%%%%%%%%%%%%%%%%%%%%%%%%%%
\section{Yukawa Unification in MSSM}
\label{sec:Yukawa_MSSM}
%%%%%%%%%%%%%%%%%%%%%%%%%%%%%%%%%%%%%%%%%%%%%%%%%%%%


\subsection{Higgs Mass}
In the MSSM, small $\tan\beta$ usually leads to small Higgs mass.
If the sfermion mass scale is very high, we can get the Higgs mass large enough.


Model point? split susy?, heavy susy?


\subsection{Threshold Correction and RGE running of Yukawa}

Although we use two or higher loop RGEs in our numerical analysis, here
we show some important one-loop beta functions to qualitatively discuss
the $b$-$\tau$ unification in our set up.  One-loop beta functions relevant
for the $b$-$\tau$ unification differ below and above the susy scale
$M_S$.  Below $M_S$, we adopt SM ($+$gaugino) beta functions,
\begin{align}
 \beta_{\tilde{y}_t} &= \frac{\tilde{y}_t}{16\pi^2} \left[ \frac{9}{2}
 \tilde{y}_t^2 + \frac{3}{2} \tilde{y}_b^2 + \tilde{y}_\tau^2 - 8g_3^2 -
 \frac{9}{4} g_2^2 - \frac{17}{20} g_1^2 \right],\\
 \beta_{\tilde{y}_b} &= \frac{\tilde{y}_b}{16\pi^2} \left[ \frac{3}{2}
 \tilde{y}_t^2 + \frac{9}{2} \tilde{y}_b^2 + \tilde{y}_\tau^2 - 8g_3^2
 - \frac{9}{4} g_2^2 - \frac{1}{4} g_1^2 \right],\\
 \beta_{\tilde{y}_\tau} &= \frac{\tilde{y}_\tau}{16\pi^2} \left[
 3\tilde{y}_t^2 + 3\tilde{y}_b^2 + \frac{5}{2} \tilde{y}_\tau^2 -
 \frac{9}{4} g_2^2 - \frac{9}{4} g_1^2 \right],
\end{align}
where $\tilde{y}_{t,b,\tau}$ are SM Yukawa couplings in the
$\overline{\rm MS}$ scheme.  On the other hand, above $M_S$, we use MSSM
beta functions,
\begin{align}
 \beta_{y_t} &= \frac{y_t}{16\pi^2} \left[ 6y_t^2 + y_b^2 - \frac{16}{3}
 g_3^2 - 3g_2^2 - \frac{13}{15} g_1^2 \right],\label{eq_beta_yt}\\
 \beta_{y_b} &= \frac{y_b}{16\pi^2} \left[ y_t^2 + 6y_b^2 + y_\tau^2 -
 \frac{16}{3} g_3^2 - 3g_2^2 - \frac{7}{15} g_1^2 \right],\label{eq_beta_yb} \\
 \beta_{y_\tau} &= \frac{y_\tau}{16\pi^2} \left[ 3y_b^2 + 4y_\tau^2 -
 3g_2^2 - \frac{9}{5} g_1^2 \right],\label{eq_beta_ytau}
\end{align}
where $y_{t,b,\tau}$ denote MSSM Yukawa couplings in the
$\overline{\rm DR}$ scheme.

At the matching scale $M_S$, Yukawa couplings $\tilde{y}$ and $y$ are
matched with each other.  Tree-level matching conditions are given by
\begin{align}
 \tilde{y}_t (M_S) &= y_t (M_S) \sin \beta,\label{eq_yt_matching}\\
 \tilde{y}_b (M_S) &= y_b (M_S) \cos \beta, \\
 \tilde{y}_\tau (M_S) &= y_\tau (M_S) \cos \beta. 
\end{align}
It is Eq.~\eqref{eq_yt_matching} and the positive sign of the first term
of Eq.~\eqref{eq_beta_yt} that explain the fact that the $y_t$ value at
the GUT scale tends to become large, sometimes about to blow up, when
$\tan \beta$ is small.  If this is the case, according to
Eq.~\eqref{eq_beta_yb} and Eq.~\eqref{eq_beta_ytau}, $y_b$ also becomes
large at the higher scale thanks to the large $y_t$, while $y_\tau$
is not enhanced.  This can affect the $b$-$\tau$ unification, as we
will mention later.

At the next-to-leading order level, Yukawa couplings, in particular
$y_b$, acquires a possibly significant effect through a threshold
correction.  The most important part of such a correction can be denoted
as
\begin{align}
 \tilde{y}_b (M_S) &= y_b (M_S) \cos \beta (1 + \Delta_b),\\
 \Delta_b &\simeq \left[ \frac{g_3^2}{6\pi^2} M_3 I(m_{\tilde{b}_1}^2,
 m_{\tilde{b}_2^2}, M_3^2) + \frac{y_t^2}{16\pi^2} A_t
 I(m_{\tilde{t}_1}^2, m_{\tilde{t}_2}^2, \mu^2) \right] \mu \tan \beta,\label{eq_delb}
\end{align}
where the loop function $I$ is defined by
\begin{align}
 I(a, b, c) \equiv - \frac{ab \ln(a/b) + bc \ln(b/c) + ca
 \ln(c/a)}{(a-b) (b-c) (c-a)}.
\end{align}
Since the correction Eq.~\eqref{eq_delb} is proportional to $\tan
\beta$, \TODO{SC: I'm now editing here!!}

%%%%%%%%%%%%%%%%%%%%%%%%%%%%%%%%%%%%%%%%%%%%%%%%%%%%
\section{Minimal Model}
\label{sec:Minimal}
%%%%%%%%%%%%%%%%%%%%%%%%%%%%%%%%%%%%%%%%%%%%%%%%%%%%
We consider the minimal SU(5) GUT model.

\subsection{Minimal Split SUSY}
Here we discuss the ``minimal" model for the split SUSY spectrum.
We assume that the SUSY breaking field $X$ is charged under some symmetry.
In this case, there is no direct coupling between the SUSY breaking field and the gauge supermultiplet like $X W^{a} W^{a}$.
The sfermion mass comes from the effective K\"ahler potential as:
\begin{align}
    K \ni -\frac{c}{M^2_{*}} X X^{\dagger} \Phi \Phi^{\dagger},
\end{align}
where $\Phi$ is matter chiral multiplets. 
With SUSY breaking effect $\langle X \rangle = F \theta^2$, the sfermion gets mass of $c |F|^2/M_*^2$.
If the cutoff scale is $M_*$ is close to the Planck scale $M_P$, the sfermion masses are comparable to the gravtino mass $m_{3/2} = {F}/{\sqrt{3}M_P}$.

The gaugino masses mainly come from the anomaly mediation effect and the thereshold correction from the Higgsino and Higgs loop.
\begin{align}
  M_1 &= \frac{3}{5} \frac{\alpha_1}{4\pi} (11 m_{3/2} + L),\\
  M_2 &= \frac{\alpha_2}{4\pi} (m_{3/2} + L),\\
  M_3 &= \frac{\alpha_3}{4\pi} (-3 m_{3/2}),
\end{align}
where $L$ represents the Higgsino-Higgs theashold corrections:
\begin{align}
      L = \mu \sin(2\beta) \frac{m_A^2}{|\mu|^2 - m_A^2} 
    \ln\frac{|\mu|^2}{m_A^2}.
\end{align}
In the present model, $L = O(m_{3/2})$ and we can tune the Wino mass light enough for the correct dark matter abundance.

\subsection{Minimal GUT}

We discuss the four dimensional minimal SUSY SU(5) GUT.
 In this theory, the SU(5) symmetry 
is broken by the VEV of $\Sigma(24)$.  The superpotential relevant 
for the GUT breaking is
%
\begin{align}
  W_\Sigma = \frac{m_\Sigma}{2} \mathrm{tr} \Sigma^2 
    + \frac{\lambda_\Sigma}{3} \mathrm{tr} \Sigma^3,
\label{eq:W_Sigma}
\end{align}
%
leading to $\langle \Sigma \rangle = v \cdot 
\mathrm{diag}(2,2,2,-3,-3)$ with $v = m_\Sigma/\lambda_\Sigma$. 
The superpotential for the Higgs fields, $H_5 = (H_C, H_u)$ 
and $\bar{H}_5 = (\bar{H}_C, H_d)$, is given by
%
\begin{align}
  W_H = \bar{H}_5 ( m_H + \lambda_H \Sigma ) H_5.
\label{eq:W_H}
\end{align}


With non-zero $\Sigma$ VEV, the vector multiplet is decomposed as
\begin{align}
 {\cal V}=\frac{1}{\sqrt{2}}
\begin{pmatrix}
 \begin{matrix}
G  -\frac{2}{\sqrt{30}} B
 \end{matrix}
&
\begin{matrix}
X^{\dagger 1} \\
X^{\dagger 2} \\
X^{\dagger 3}
\end{matrix}
&
\begin{matrix}
 Y^{\dagger 1}\\ Y^{\dagger 2} \\ Y^{\dagger 3}
\end{matrix}
\\
\begin{matrix}
 X_1 & X_2 & X_3 \\
 Y_1 & Y_2 & Y_3  
\end{matrix}
&
\begin{matrix}
 \frac{1}{\sqrt{2}}W^3+\frac{3}{\sqrt{30}}B \\ W^-
\end{matrix}
&
\begin{matrix}
W^+ \\ - \frac{1}{\sqrt{2}}W^3+\frac{3}{\sqrt{30}}B
\end{matrix}
\end{pmatrix}
~.
\end{align} 
The mass of $X$ gauge multiplet $ (-5/6, 2, 3)+ (5/6, 2, \bar{3})$ is
\begin{align}
m_X = 5 \sqrt{5} g_5 v.
\end{align}
The adjoint $\Sigma$ is decomposed into Goldstone, SM adjoint  $(0,1,8), (0,3,1)$ and singlet multiplets.
\begin{align}
\Sigma - \langle \Sigma  \rangle =
\begin{pmatrix}
 \Sigma_{8}& \frac{1}{\sqrt{2}}\Gamma_{(3,2)} \\
 \frac{1}{\sqrt{2}}\Gamma^{c}_{(3^*,2)} & \Sigma_3
\end{pmatrix}
+\frac{1}{2\sqrt{15}}
\begin{pmatrix}
 2&0\\0&-3
\end{pmatrix}
\Sigma_{1}~.
\end{align}
 The masses are 
\begin{align}
  m_{\Sigma_8} =  m_{\Sigma_3} = \frac{5}{2}\lambda_{\Sigma} v
\qquad
  m_{\Sigma_1} = \frac{1}{2}\lambda_{\Sigma} v
\label{eq:mu-GUT}
\end{align}

%
With the $\Sigma$ VEV, the supersymmetric masses for the 
MSSM Higgs doublets $H_{u,d}$ and their GUT partners $H_C, \bar{H}_C$:
%
\begin{align}
  \mu = m_H - 3 \lambda_H v,
\qquad
  \mu_C = m_H + 2 \lambda_H v.
\label{eq:mu-GUT}
\end{align}


\subsubsection{GUT and Matching}
At one loop level, relations between MSSM and GUT gauge couplings are
\begin{align}
 \frac{1}{g_1^2(Q)}&=\frac{1}{g_5^2(Q)}+\frac{1}{8\pi^2}\biggl[
\frac{2}{5}
\ln \frac{Q}{M_{H_C}}-10\ln\frac{Q}{M_X}
\biggr]~,\nonumber \\
 \frac{1}{g_2^2(Q)}&=\frac{1}{g_5^2(Q)}+\frac{1}{8\pi^2}\biggl[
2\ln \frac{Q}{M_\Sigma}-6\ln\frac{Q}{M_X}
\biggr]~,\nonumber \\
 \frac{1}{g_3^2(Q)}&=\frac{1}{g_5^2(Q)}+\frac{1}{8\pi^2}\biggl[
\ln \frac{Q}{M_{H_C}}+3\ln \frac{Q}{M_\Sigma}-4\ln\frac{Q}{M_X}
\biggr]~.
\end{align}
Here $g_{1,2,3}$ are the MSSM $\overline{\rm DR}$ gauge couplings and $g_5$ is the
GUT $\overline{\rm DR}$ gauge coupling, which are mass-independently defined.
From the aligns we have
\begin{align}
\sqrt{14} R_H \equiv \frac{3}{g_2^2(Q)}- \frac{2}{g_3^2(Q)}- \frac{1}{g_1^2(Q)}
&=-\frac{3}{10\pi^2}\ln \frac{Q}{M_{H_C}}~, \nonumber \\
\sqrt{32} R_X \equiv \frac{5}{g_1^2(Q)}- \frac{3}{g_2^2(Q)}- \frac{2}{g_3^2(Q)}
&=-\frac{3}{2\pi^2}\ln \frac{Q^3}{M_X^2M_{\Sigma}}~.
\label{conditions}
\end{align}
The zero points predict the mass scale of GUT particles.

\subsubsection{With higher dimensional contributions}
The GUT scale (maybe  $\sim 10^{16-17}$ GeV) is close to the cut-off scale e.g., the Planck scale $M_P = 2\times 10^{18}$ GeV.
In this case, we may expect the corrections from higher dimensional operators also contribute the gauge couplings.
The dominant contributions come from 
\begin{align}
  {\cal L} \supset \frac{1}{2 g^2} \int d^2\theta\, 
      \left\{ {\rm Tr}[{\cal W}^\alpha {\cal W}_\alpha] 
    + \frac{a}{\Lambda} {\rm Tr}[\Sigma {\cal W}^\alpha {\cal W}_\alpha] 
    + \frac{b}{\Lambda^2}{\rm Tr}[\Sigma {\cal W}^\alpha ] {\rm Tr}[\Sigma {\cal W}_\alpha ]
     \right\}
    + {\rm h.c.},
\label{eq:threshold}
\end{align}
With developing $\Sigma$ VEV, the relation of the gauge couplings are modified,
\begin{align}
\Delta R_H = -\frac{6\sqrt{2}a}{\sqrt{7}g_5^2} \frac{v}{\Lambda} \simeq 1.1 a \times \frac{m_X}{\Lambda}
,\qquad
\Delta R_X = \frac{20\sqrt{5}b}{ \sqrt{38}g_5^2} \frac{v^2}{\Lambda^2}
\simeq 0.2 b \times\frac{m_X^2}{\Lambda^2}
\end{align}
The leading dimension-five operator (the second term above) gives 
a correction to $R_H$, but not to $R_X$. The $R_X$ is corrected only at 
order $\langle \Sigma \rangle^2/\Lambda^2$, which is small.  We can, 
therefore, read off the mass of the $XY$ gauge boson, $M_X \approx 
\langle \Sigma \rangle$, by the points of $R_X$,
%
\begin{align}
  R_X(M_X) \approx 0.
\label{eq:R_X-M_X}
\end{align}
%
On the other hand, since $R_H$ receives a relatively large correction from 
the dimension-five operator, it does not strongly constrain $M_{H_C}$---any 
value of $M_{H_C}$ is consistent as long as $R_H$ at that scale is reasonably 
small
%
\begin{align}
  \left| R_H(M_{H_C}) \right| 
  \approx \left| -\frac{a v}{\Lambda} \right| 
\label{eq:R_H-M_H}
\end{align}
%


%%%%%%%%%%%%%%%%%%%%%%%%%%%%%%%%%%%%%%%%
\section{Conclusion and discussion}
\label{sec:conclusion}
%%%%%%%%%%%%%%%%%%%%%%%%%%%%%%%%%%%%%%%%










%%%%%%%%%%%%%%%%%%%%%%%%%%%%%%%%%%%%
\section*{Acknowledgments}
%%%%%%%%%%%%%%%%%%%%%%%%%%%%%%%%%%%%
This work is supported by Grant-in-Aid for Scientific Research from the
Ministry of Education, Culture, Sports, Science, and Technology (MEXT),
Japan, No. 17H02878 and 18K13535 (S.S.) and by World Premier
International Research Center Initiative (WPI), MEXT, Japan (S.S.).
This work was also supported by JSPS KAKENHI Grant, No. 17J00813 (S.C)


%%%%%%%%%%%%%%%%%%%%%%%%%%%%%%%%%%%%%%%%%%%%%%
%\section*{Appendix}
%\appendix
%%%%%%%%%%%%%%%%%%%%%%%%%%%%%%%%%%%%%%%%%%%%%


%%%%%%%%%%%%% References %%%%%%%%%%%%%%%%%%%
\bibliographystyle{aps}
\bibliography{ref}
%%%%%%%%%%%%%%%%%%%%%%%%%%%%%%%%%%%%%%%%%%%%

\end{document}
